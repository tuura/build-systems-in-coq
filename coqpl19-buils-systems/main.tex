%% For double-blind review submission, w/o CCS and ACM Reference (max submission space)
% \documentclass[sigplan,review,anonymous]{acmart}\settopmatter{printfolios=true,printccs=false,printacmref=false}
%% For double-blind review submission, w/ CCS and ACM Reference
%\documentclass[sigplan,review,anonymous]{acmart}\settopmatter{printfolios=true}
% For single-blind review submission, w/o CCS and ACM Reference (max submission space)
\documentclass[sigplan,review]{acmart}\settopmatter{printfolios=true,printccs=false,printacmref=false}
%% For single-blind review submission, w/ CCS and ACM Reference
%\documentclass[sigplan,review]{acmart}\settopmatter{printfolios=true}
%% For final camera-ready submission, w/ required CCS and ACM Reference
%\documentclass[sigplan]{acmart}\settopmatter{}

\usepackage{bookmark}
\usepackage{booktabs}
\usepackage{subcaption}
\usepackage[utf8]{inputenc}
\usepackage[T1]{fontenc}
\usepackage{xspace}
\usepackage{fancyhdr}

% Haskell code snippets and useful shortcuts
\usepackage{minted}
\setminted[coq]{escapeinside=@@}
\newcommand{\hs}{\mintinline{haskell}}
\newcommand{\coq}{\mintinline{coq}}
\newcommand{\cmd}[1]{\textsf{\color[rgb]{0,0,0.5} #1}}
\newcommand{\teq}{\smaller $\sim$}
\newcommand{\ghci}{$\lambda$>}
\newcommand{\defeq}{\stackrel{\text{def}}{=}}
\newcommand{\std}[1]{{\color[rgb]{0,0.3,0} #1}}
\newcommand{\blk}[1]{{\color[rgb]{0,0,0} #1}}

\newcommand{\Bazel}{\textsc{Bazel}\xspace}
\newcommand{\Buck}{\textsc{Buck}\xspace}
\newcommand{\Calc}{\textsc{Calc}\xspace}
\newcommand{\Cloud}{\textsc{Cloud}\xspace}
\newcommand{\CloudBuild}{\textsc{CloudBuild}\xspace}
\newcommand{\Dune}{\textsc{Dune}\xspace}
\newcommand{\Excel}{\textsc{Excel}\xspace}
\newcommand{\Fabricate}{\textsc{Fabricate}\xspace}
\newcommand{\Incremental}{\textsc{Incremental}\xspace}
\newcommand{\Make}{\textsc{Make}\xspace}
\newcommand{\Ninja}{\textsc{Ninja}\xspace}
\newcommand{\Nix}{\textsc{Nix}\xspace}
\newcommand{\Pluto}{\textsc{Pluto}\xspace}
\newcommand{\Redo}{\textsc{Redo}\xspace}
\newcommand{\Reflow}{\textsc{Reflow}\xspace}
\newcommand{\Shake}{\textsc{Shake}\xspace}
\newcommand{\Tup}{\textsc{Tup}\xspace}
\newcommand{\store}{\hs{k}~\hs{->}~\hs{v}\xspace}
\newcommand{\storef}{\hs{k}~\hs{->}~\hs{f}~\hs{v}\xspace}

%% Conference information
%% Supplied to authors by publisher for camera-ready submission;
%% use defaults for review submission.
\acmConference[CoqPL'19]{The Fifth International Workshop on Coq for Programming Languages}{January 19, 2019}{Cascais/Lisbon, Portugal}
\acmYear{2019}
\acmISBN{} % \acmISBN{978-x-xxxx-xxxx-x/YY/MM}
\acmDOI{} % \acmDOI{10.1145/nnnnnnn.nnnnnnn}
\startPage{1}

%% Copyright information
%% Supplied to authors (based on authors' rights management selection;
%% see authors.acm.org) by publisher for camera-ready submission;
%% use 'none' for review submission.
\setcopyright{none}
%\setcopyright{acmcopyright}
%\setcopyright{acmlicensed}
%\setcopyright{rightsretained}
%\copyrightyear{2018}           %% If different from \acmYear

%% Bibliography style
\bibliographystyle{ACM-Reference-Format}
%% Citation style
%\citestyle{acmauthoryear}  %% For author/year citations
%\citestyle{acmnumeric}     %% For numeric citations
%\setcitestyle{nosort}      %% With 'acmnumeric', to disable automatic
                            %% sorting of references within a single citation;
                            %% e.g., \cite{Smith99,Carpenter05,Baker12}
                            %% rendered as [14,5,2] rather than [2,5,14].
%\setcitesyle{nocompress}   %% With 'acmnumeric', to disable automatic
                            %% compression of sequential references within a
                            %% single citation;
                            %% e.g., \cite{Baker12,Baker14,Baker16}
                            %% rendered as [2,3,4] rather than [2-4].


%%%%%%%%%%%%%%%%%%%%%%%%%%%%%%%%%%%%%%%%%%%%%%%%%%%%%%%%%%%%%%%%%%%%%%
%% Note: Authors migrating a paper from traditional SIGPLAN
%% proceedings format to PACMPL format must update the
%% '\documentclass' and topmatter commands above; see
%% 'acmart-pacmpl-template.tex'.
%%%%%%%%%%%%%%%%%%%%%%%%%%%%%%%%%%%%%%%%%%%%%%%%%%%%%%%%%%%%%%%%%%%%%%


%% Some recommended packages.
\usepackage{booktabs}   %% For formal tables:
                        %% http://ctan.org/pkg/booktabs
\usepackage{subcaption} %% For complex figures with subfigures/subcaptions
                        %% http://ctan.org/pkg/subcaption


\begin{document}

%% Title information
\title[]{A Coq Formalisation of Build Systems}         %% [Short Title] is optional;
                                        %% when present, will be used in
                                        %% header instead of Full Title.
% \titlenote{with title note}             %% \titlenote is optional;
                                        %% can be repeated if necessary;
                                        %% contents suppressed with 'anonymous'
\subtitle{Experience report}                     %% \subtitle is optional
% \subtitlenote{with subtitle note}       %% \subtitlenote is optional;
                                        %% can be repeated if necessary;
                                        %% contents suppressed with 'anonymous'

%% Author information
%% Contents and number of authors suppressed with 'anonymous'.
%% Each author should be introduced by \author, followed by
%% \authornote (optional), \orcid (optional), \affiliation, and
%% \email.
%% An author may have multiple affiliations and/or emails; repeat the
%% appropriate command.
%% Many elements are not rendered, but should be provided for metadata
%% extraction tools.

\author{Author names}
\affiliation{
  \institution{Removed for blind review}
}

% \author{Georgy Lukyanov}
% \affiliation{
%   \department{School of Engineering}
%   \institution{Newcastle University}
%   \city{Newcastle upon Tyne}
%   \country{United Kingdom}
% }
% \email{g.lukyanov2@ncl.ac.uk}

% \author{Andrey Mokhov}
% \affiliation{
%   \department{School of Engineering}
%   \institution{Newcastle University}
%   \city{Newcastle upon Tyne}
%   \country{United Kingdom}
% }
% \email{andrey.mokhov@ncl.ac.uk}

%% Abstract
%% Note: \begin{abstract}...\end{abstract} environment must come
%% before \maketitle command
\begin{abstract}
The gene pool of software build systems and various incremental computation
frameworks is becoming richer every day. These complex build systems and
frameworks use subtle algorithms and are mission-critical, yet to the best of
our knowledge they come with no formal proofs of correctness.

%  are no formal verification
%  their correctness has not
% there is no machine-checked validation of their correctness.

A recent ICFP paper ``Build Systems \`a la Carte'' presented a definition of
correctness for build systems, and modelled several major build systems in
Haskell, without exhibiting any proof of their correctness. We build on this
work by translating the Haskell abstractions to Coq and making the necessary
adjustments to capture the notion of build task acyclicity, which is essential
for proving termination.

This is an experience report on on-going work which is very far from being
complete. We present our motivation and key abstractions developed so far, and
seek feedback and collaboration from the Coq community.
\end{abstract}

%% 2012 ACM Computing Classification System (CSS) concepts
%% Generate at 'http://dl.acm.org/ccs/ccs.cfm'.
\begin{CCSXML}
<ccs2012>
<concept>
<concept_id>10011007.10011006.10011008</concept_id>
<concept_desc>Software and its engineering~General programming languages</concept_desc>
<concept_significance>500</concept_significance>
</concept>
<concept>
<concept_id>10003456.10003457.10003521.10003525</concept_id>
<concept_desc>Social and professional topics~History of programming languages</concept_desc>
<concept_significance>300</concept_significance>
</concept>
</ccs2012>
\end{CCSXML}

\ccsdesc[500]{Software and its engineering~General programming languages}
\ccsdesc[300]{Social and professional topics~History of programming languages}
%% End of generated code


%% Keywords
%% comma separated list
\keywords{build systems, formalisation}  %% \keywords are mandatory in final camera-ready submission

%% \maketitle
%% Note: \maketitle command must come after title commands, author
%% commands, abstract environment, Computing Classification System
%% environment and commands, and keywords command.
\maketitle

\vspace{-3mm}
\section{Introduction}
\vspace{-1mm}

Build systems, such as \Make~\cite{feldman1979make},
\Bazel~\cite{bazel} and \Shake~\cite{mitchell2012shake}, are used by every
software developer on the planet, but their subtle algorithms have not yet been
scrutinised using formal methods. Detailed and executable models of major build
systems can be found in~\cite{Mokhov2018icfp}, however, these models have not
been formally verified.

We believe that developing a formal verification framework for build systems
is a useful, achievable and interesting goal. Modern build systems are
very difficult to test; they use many optimisations whose correctness is not
obvious; they drown in accidental complexity that hides a rich internal
structure --- we hope to address all these issues by looking at build systems
through the lens of formal methods, and, more concretely, by modelling build
systems in Coq.

What is a build system? Below we briefly recite main build systems notions
using the vocabulary from~\cite{Mokhov2018icfp}.

Build systems operate on \emph{key/value stores}, where the keys are typically
filenames and the values are file contents. Some values must be provided by
the user as \emph{input}, e.g. by editing \emph{source files}, whereas
\emph{output} values are produced by the build system by executing \emph{build
tasks} (also called \emph{build rules}).

As well as the key/value mapping, the store also contains \emph{build
information} maintained by the build system itself, which is persistently stored
between build runs.

To run a build system, the user should specify how to compute the values for
output keys using the values of its \emph{dependencies}. We call this
specification the \emph{task description}.

A \emph{build system} takes a task description, a \emph{target} key, and a
store, and returns a new store in which the target key and all its dependencies
have an up to date value.

% The next Section~\ref{sec-abstractions} formalises these notions in Coq,
% followed by a brief discussion in Section~\ref{sec-discussion}.

\vspace{-3mm}
\section{Encoding build systems in Coq}\label{sec-abstractions}

\begin{figure*}[t]
\begin{minted}{coq}
(* An abstract store containing a key/value map and persistent build information *)
Inductive Store (I K V : Type) := mkStore { info : I; values : K -> V }.
\end{minted}
\vspace{-1mm}
\begin{minted}{coq}
(* A Task describes how to build values of type V given a way to find values of its dependencies *)
Inductive Task (C : (Type -> Type) -> Type) (K V : Type) := {
  run : forall {F} `{CF: C F}, (K -> F V) -> F V }.
\end{minted}
\vspace{-1mm}
\begin{minted}{coq}
(* Tasks associates a Task to every output key; input keys are associated with Nothing. *)
Definition Tasks (C : (Type -> Type) -> Type) (K V : Type) := K -> Maybe (Task C K V).
\end{minted}
\vspace{-1mm}
\begin{minted}{coq}
(* Given a task description and a target key, a build system transforms the store. *)
Definition Build (C : (Type -> Type) -> Type) (I K V : Type) :=
  Tasks C K V -> K -> Store I K V -> Store I K V.
\end{minted}
\vspace{-1mm}
\begin{minted}{coq}
(* A task in AcyclicTasks can only depend on keys that are smaller than its own key. *)
Definition AcyclicTasks (C : (Type -> Type) -> Type) (V : Type):=
  forall (k : nat), Maybe (Task C (Fin.t k) V).
\end{minted}
% \vspace{-1mm}
% \begin{minted}{coq}
% (* Given a acyclic task description, a target key, and a store, the acyclic build
%  * system returns a new store and is guarantied to terminates. *)
% Definition AcyclicBuild (C : (Type -> Type) -> Type) (I : Type) :=
%   AcyclicTasks C -> K -> Store I K V -> Store I K V.
% \end{minted}
\vspace{-4mm}
\caption{Build system abstractions in Coq.}\label{fig-defs}
\vspace{-4mm}
\end{figure*}

Following the approach in~\cite{Mokhov2018icfp}, we model \hs{Store} as
an abstract data type parameterised by the type of keys~\hs{K}, values~\hs{V},
and build information~\hs{I} (see the Fig.~\ref{fig-defs}). A \hs{Task}
describes how to compute a value of a key (\hs{F V}), given a way to find values
of its dependencies (the callback \hs{K -> F V}). The type of effect~\hs{F}
associated with computing values is constrained by~\hs{C}, which emulates
Haskell's kind \hs{Constraint} and can be instantiated with concrete constraints,
such as \hs{Applicative} (for statically known task dependencies, e.g. as in \Make)
and \hs{Monad} (for dynamic dependencies, which become known only during the
execution of a task, as in \Shake).

A \hs{Tasks} associates a task to a every output key, and \hs{Nothing} to input
keys. Just like a single \hs{Task}, the compound task description \hs{Tasks} is
polymorphic over the effect type~\hs{F}. It is very important to have the
flexibility of interpreting the very same task description in
different~\hs{F}'s: (i) a build system typically fixes \hs{F} to some state
monad in order to side-effect the \hs{Store} during the build; (ii) furthermore,
to extract accurate \emph{task dependencies}, a task can be executed in a
\emph{tracking} \hs{Applicative} (e.g. \hs{Const}) or \hs{Monad}
(e.g. \hs{Writer}).

As an example, consider the following task description for computing generalised
Fibonacci sequence:

\vspace{-1mm}
\begin{minted}{coq}
Definition fib : Tasks Applicative nat nat :=
  fun n =>
  match n with
  | 0  => Nothing
  | 1  => Nothing
  | S (S m) => Just {|
               run := fun _ _ => fun fetch =>
                      Nat.add <$> fetch (S m)
                              <*> fetch m |}
  end.
\end{minted}
\vspace{-1mm}

\noindent
The keys~\coq{0} and~\coq{1} are inputs (i.e. the user may choose
how to initialise the first two elements of the generalised Fibonacci sequence),
and are therefore associated to \hs{Nothing}. All other keys are associated to
non-trivial tasks that, given the callback \hs{fetch}, use it to find the values
of the two previous Fibonacci numbers in the sequence and return their sum.

\noindent
Dependencies of applicative tasks can be calculated
statically~\cite{free-applicatives}, without providing any actual values:

\vspace{-1mm}
\begin{minted}{coq}
Definition dependencies {K V : Type}
  (task : Maybe (Task Applicative K V)) : list K :=
  match task with
  | Nothing => nil
  | Just act => getConst
    ((run act) (fun k => mkConst (cons k nil)))
  end.
\end{minted}
\vspace{-1mm}

\noindent
The function~\hs{dependencies} instantiates~\hs{F}
to~\hs{Const}~\hs{(}\hs{list}~\hs{K)} and executes the task with a~\hs{fetch}
callback that just records the fetched key. The keys then get accumulated to a
list by the \hs{Applicative} instance of the data
type~\hs{Const}~\hs{(}\hs{list}~\hs{K)}.

The task description~\hs{fib} is applicative, and can be statically analysed.
The correctness of the function \hs{dependencies} is trivially proved in this
specific case:

\vspace{-1mm}
\begin{minted}{coq}
Theorem dependencies_fib_correct : forall n,
  dependencies (fib (S (S n))) = (S n@\,@::@\,@n@\,@::@\,@nil).
Proof. reflexivity. Qed.
\end{minted}
\vspace{-1mm}

Task descriptions trap the recursion inside the \hs{fetch} callback. In case of
the function~\hs{dependencies}, the callback is not recursive, and the function
always terminates. However, build systems described in~\cite{Mokhov2018icfp}
require a \emph{recursive callback}: indeed, to find the value of a dependency we
may need to rebuild it, which may trigger further recursive rebuilds.
Termination is no longer guaranteed in such settings. To tackle this issue, we
introduce the notion of \emph{acyclic task descriptions}. In contrast to
possibly cyclic \hs{Tasks}, a task that belongs to \hs{AcyclicTasks}
can depend only on keys that are smaller than its own key. Below we re-define
\hs{fib} as an \hs{AcyclicTasks}:

\vspace{-1mm}
\begin{minted}{coq}
Definition@\,@fib@\,@:@\,@AcyclicTasks@\,@Applicative@\,@nat:=@\,@fun n =>
  match n with
  | 0  => Nothing
  | 1  => Nothing
  | S (S m) => Just
      {| run := fun _ _ => fun fetch =>
         Nat.add <$> fetch (from_nat (S m))
                 <*> fetch (inject1 (from_nat m)) |}
  end.
\end{minted}
\vspace{-1mm}

\noindent
The \hs{fetch} callback corresponding to the key \hs{k} now expects a bounded
natural number of type~\hs{Fin.t}~\hs{k}, which requires \emph{dependent types},
as seen in the type of \hs{AcyclicTasks} in Fig.~\ref{fig-defs}(bottom). Armed
with acyclic tasks, we can prove termination and correctness of build systems
(on-going work).

% thus we coerce the subterms~\coq{m} and~\coq{S m} and supply those as arguments
% to~\hs{fetch}.

\vspace{-2mm}
\section{Discussion and Future Work}\label{sec-discussion}
\vspace{-1mm}

The Haskell-embedded framework for modelling build systems presented
in~\cite{Mokhov2018icfp} relies on advanced GHC extensions such
as~\emph{Rank2Types} and~\emph{ConstraintKinds} in a crucial way for defining
key properties and components of build systems in a high-level and reusable API.
As we have seen, Coq's expressive power is sufficient for encoding this API;
furthermore, Coq provides additional features, such as dependent types, that
allow us make correctness proofs about the API.

The essence of every build system lies in its specific \hs{fetch} callback. The
simplest build system from~\cite{Mokhov2018icfp}, \hs{busy}, defines \hs{fetch}
as a recursive function in the~\hs{State}~\hs{(Store}~\hs{()}~\hs{K}~\hs{V)}
monad and is not accepted by Coq's termination checker if encoded naively. More
sophisticated build systems involve further complexity related to achieving
minimality (never executing unnecessary tasks), dynamic dependencies (\Shake),
and cloud builds (\Bazel). Our goal is to prove that the models of these
build systems presented in~\cite{Mokhov2018icfp} are correct and, crucially,
\emph{to do that in a way that does not sacrifice the simplicity of these models}.
As non experts in Coq, we have not yet been able to achieve this goal. We hope
that the Coq community can steer us towards the right abstractions and tools.

% Bearing all that in mind, the main take-away from this experience is the
% necessity to rigorously prove termination of a build system before even
% attempting to prove its correctness.

%% Bibliography
\bibliography{biblio}

\end{document}
