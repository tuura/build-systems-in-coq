%% For double-blind review submission, w/o CCS and ACM Reference (max submission space)
% \documentclass[sigplan,review,anonymous]{acmart}\settopmatter{printfolios=true,printccs=false,printacmref=false}
%% For double-blind review submission, w/ CCS and ACM Reference
%\documentclass[sigplan,review,anonymous]{acmart}\settopmatter{printfolios=true}
% For single-blind review submission, w/o CCS and ACM Reference (max submission space)
\documentclass[sigplan,review]{acmart}\settopmatter{printfolios=true,printccs=false,printacmref=false}
%% For single-blind review submission, w/ CCS and ACM Reference
%\documentclass[sigplan,review]{acmart}\settopmatter{printfolios=true}
%% For final camera-ready submission, w/ required CCS and ACM Reference
%\documentclass[sigplan]{acmart}\settopmatter{}

\usepackage{bookmark}
\usepackage{booktabs}
\usepackage{subcaption}
\usepackage[utf8]{inputenc}
\usepackage[T1]{fontenc}
\usepackage{xspace}
\usepackage{fancyhdr}

% Haskell code snippets and useful shortcuts
\usepackage{minted}
\setminted[coq]{escapeinside=@@}
\newcommand{\hs}{\mintinline{haskell}}
\newcommand{\coq}{\mintinline{coq}}
\newcommand{\cmd}[1]{\textsf{\color[rgb]{0,0,0.5} #1}}
\newcommand{\teq}{\smaller $\sim$}
\newcommand{\ghci}{$\lambda$>}
\newcommand{\defeq}{\stackrel{\text{def}}{=}}
\newcommand{\std}[1]{{\color[rgb]{0,0.3,0} #1}}
\newcommand{\blk}[1]{{\color[rgb]{0,0,0} #1}}

\newcommand{\Bazel}{\textsc{Bazel}\xspace}
\newcommand{\Buck}{\textsc{Buck}\xspace}
\newcommand{\Calc}{\textsc{Calc}\xspace}
\newcommand{\Cloud}{\textsc{Cloud}\xspace}
\newcommand{\CloudBuild}{\textsc{CloudBuild}\xspace}
\newcommand{\Dune}{\textsc{Dune}\xspace}
\newcommand{\Excel}{\textsc{Excel}\xspace}
\newcommand{\Fabricate}{\textsc{Fabricate}\xspace}
\newcommand{\Incremental}{\textsc{Incremental}\xspace}
\newcommand{\Make}{\textsc{Make}\xspace}
\newcommand{\Ninja}{\textsc{Ninja}\xspace}
\newcommand{\Nix}{\textsc{Nix}\xspace}
\newcommand{\Pluto}{\textsc{Pluto}\xspace}
\newcommand{\Redo}{\textsc{Redo}\xspace}
\newcommand{\Reflow}{\textsc{Reflow}\xspace}
\newcommand{\Shake}{\textsc{Shake}\xspace}
\newcommand{\Tup}{\textsc{Tup}\xspace}
\newcommand{\store}{\hs{k}~\hs{->}~\hs{v}\xspace}
\newcommand{\storef}{\hs{k}~\hs{->}~\hs{f}~\hs{v}\xspace}

%% Conference information
%% Supplied to authors by publisher for camera-ready submission;
%% use defaults for review submission.
\acmConference[CoqPL'19]{The Fifth International Workshop on Coq for Programming Languages}{January 19, 2019}{Cascais/Lisbon, Portugal}
\acmYear{2019}
\acmISBN{} % \acmISBN{978-x-xxxx-xxxx-x/YY/MM}
\acmDOI{} % \acmDOI{10.1145/nnnnnnn.nnnnnnn}
\startPage{1}

%% Copyright information
%% Supplied to authors (based on authors' rights management selection;
%% see authors.acm.org) by publisher for camera-ready submission;
%% use 'none' for review submission.
\setcopyright{none}
%\setcopyright{acmcopyright}
%\setcopyright{acmlicensed}
%\setcopyright{rightsretained}
%\copyrightyear{2018}           %% If different from \acmYear

%% Bibliography style
\bibliographystyle{ACM-Reference-Format}
%% Citation style
%\citestyle{acmauthoryear}  %% For author/year citations
%\citestyle{acmnumeric}     %% For numeric citations
%\setcitestyle{nosort}      %% With 'acmnumeric', to disable automatic
                            %% sorting of references within a single citation;
                            %% e.g., \cite{Smith99,Carpenter05,Baker12}
                            %% rendered as [14,5,2] rather than [2,5,14].
%\setcitesyle{nocompress}   %% With 'acmnumeric', to disable automatic
                            %% compression of sequential references within a
                            %% single citation;
                            %% e.g., \cite{Baker12,Baker14,Baker16}
                            %% rendered as [2,3,4] rather than [2-4].


%%%%%%%%%%%%%%%%%%%%%%%%%%%%%%%%%%%%%%%%%%%%%%%%%%%%%%%%%%%%%%%%%%%%%%
%% Note: Authors migrating a paper from traditional SIGPLAN
%% proceedings format to PACMPL format must update the
%% '\documentclass' and topmatter commands above; see
%% 'acmart-pacmpl-template.tex'.
%%%%%%%%%%%%%%%%%%%%%%%%%%%%%%%%%%%%%%%%%%%%%%%%%%%%%%%%%%%%%%%%%%%%%%


%% Some recommended packages.
\usepackage{booktabs}   %% For formal tables:
                        %% http://ctan.org/pkg/booktabs
\usepackage{subcaption} %% For complex figures with subfigures/subcaptions
                        %% http://ctan.org/pkg/subcaption


\begin{document}

%% Title information
\title[]{A Coq Formalisation of Build Systems}         %% [Short Title] is optional;
                                        %% when present, will be used in
                                        %% header instead of Full Title.
% \titlenote{with title note}             %% \titlenote is optional;
                                        %% can be repeated if necessary;
                                        %% contents suppressed with 'anonymous'
\subtitle{Experience report}                     %% \subtitle is optional
% \subtitlenote{with subtitle note}       %% \subtitlenote is optional;
                                        %% can be repeated if necessary;
                                        %% contents suppressed with 'anonymous'

%% Author information
%% Contents and number of authors suppressed with 'anonymous'.
%% Each author should be introduced by \author, followed by
%% \authornote (optional), \orcid (optional), \affiliation, and
%% \email.
%% An author may have multiple affiliations and/or emails; repeat the
%% appropriate command.
%% Many elements are not rendered, but should be provided for metadata
%% extraction tools.


\author{Georgy Lukyanov}
\affiliation{
  \department{School of Engineering}
  \institution{Newcastle University}
  \city{Newcastle upon Tyne}
  \country{United Kingdom}
}
\email{g.lukyanov2@ncl.ac.uk}

\author{Andrey Mokhov}
\affiliation{
  \department{School of Engineering}
  \institution{Newcastle University}
  \city{Newcastle upon Tyne}
  \country{United Kingdom}
}
\email{andrey.mokhov@ncl.ac.uk}

%% Abstract
%% Note: \begin{abstract}...\end{abstract} environment must come
%% before \maketitle command
\begin{abstract}
The gene pool of software build systems and various incremental computation
frameworks is becoming richer every day. These complex build systems and
frameworks use subtle algorithms and are mission-critical, yet to the best of
our knowledge they come with no formal proofs of correctness.

%  are no formal verification
%  their correctness has not
% there is no machine-checked validation of their correctness.

A recent ICFP paper ``Build Systems \`a la Carte'' presented a definition of
correctness for build systems, and modelled several major build systems in
Haskell, without exhibiting any proof of their correctness. We build on this
work by translating the Haskell abstractions to Coq and making the necessary
adjustments to capture the notion of build task acyclicity, which is essential
for proving termination.

This is an experience report on on-going work which is very far from being
complete. We present our motivation and key abstractions developed so far, and
seek feedback and collaboration from the Coq community.
\end{abstract}

%% 2012 ACM Computing Classification System (CSS) concepts
%% Generate at 'http://dl.acm.org/ccs/ccs.cfm'.
\begin{CCSXML}
<ccs2012>
<concept>
<concept_id>10011007.10011006.10011008</concept_id>
<concept_desc>Software and its engineering~General programming languages</concept_desc>
<concept_significance>500</concept_significance>
</concept>
<concept>
<concept_id>10003456.10003457.10003521.10003525</concept_id>
<concept_desc>Social and professional topics~History of programming languages</concept_desc>
<concept_significance>300</concept_significance>
</concept>
</ccs2012>
\end{CCSXML}

\ccsdesc[500]{Software and its engineering~General programming languages}
\ccsdesc[300]{Social and professional topics~History of programming languages}
%% End of generated code


%% Keywords
%% comma separated list
\keywords{build systems, formalisation}  %% \keywords are mandatory in final camera-ready submission

%% \maketitle
%% Note: \maketitle command must come after title commands, author
%% commands, abstract environment, Computing Classification System
%% environment and commands, and keywords command.
\maketitle

\vspace{-3mm}
\section{Introduction}
\vspace{-1mm}

Build systems, such as \Make~\cite{feldman1979make},
\Bazel~\cite{bazel} and \Shake~\cite{mitchell2012shake}, are used by every
software developer on the planet, but their subtle algorithms have not yet been
scrutinised using formal methods. Detailed and executable models of major build
systems can be found in~\cite{Mokhov2018icfp}, however, these models have not
been formally verified.

We believe that developing a formal verification framework for build systems
is a useful, achievable and interesting goal. Modern build systems are
very difficult to test; they use many optimisations whose correctness is not
obvious; they drown in accidental complexity that hides a rich internal
structure --- we hope to address all these issues by looking at build systems
through the lens of formal methods, and, more concretely, by modelling build
systems in Coq.

What is a build system? Below we briefly recite main build systems notions
using the vocabulary from~\cite{Mokhov2018icfp}.

Build systems operate on \emph{key/value stores}, where the keys are typically
filenames and the values are file contents. Some values must be provided by
the user as \emph{input}, e.g. by editing \emph{source files}, whereas
\emph{output} values are produced by the build system by executing \emph{build
tasks} (also called \emph{build rules}).

As well as the key/value mapping, the store also contains \emph{build
information} maintained by the build system itself, which is persistently stored
between build runs.

To run a build system, the user should specify how to compute the values for
output keys using the values of its \emph{dependencies}. We call this
specification the \emph{task description}.

A \emph{build system} takes a task description, a \emph{target} key, and a
store, and returns a new store in which the target key and all its dependencies
have an up to date value.

The next Section~\ref{sec-abstractions} formalises these notions in Coq,
followed by a brief discussion in Section~\ref{sec-discussion}.

\vspace{-3mm}
\section{Encoding build systems in Coq}\label{sec-abstractions}
\vspace{-1mm}

\begin{figure*}[t]
\begin{minted}{coq}
(* An abstract store containing a key/value map and persistent build information *)
Inductive Store (I K V : Type) := mkStore { info : I; values : K -> V}.
\end{minted}
\vspace{-1mm}
\begin{minted}{coq}
(* Task describes how to build values of type V from keys of type K *)
Inductive Task (C : (Type -> Type) -> Type) (K V : Type) := {
  run : forall {F} `{CF: C F}, (K -> F V) -> F V}.
\end{minted}
\vspace{-1mm}
\begin{minted}{coq}
(* Tasks associates a Task to every non-input key; input keys are associated with Nothing. *)
Definition Tasks (C : (Type -> Type) -> Type) (K V : Type) :=
  K -> Maybe (Task C K V).
\end{minted}
\vspace{-1mm}
\begin{minted}{coq}
(* Given a task description, a target key, and a store, the build
 * system returns a new store in which the value of the target key is up to date. *)
Definition Build (C : (Type -> Type) -> Type) (I K V : Type) :=
  Tasks C K V -> K -> Store I K V -> Store I K V.
\end{minted}
\vspace{-1mm}
\begin{minted}{coq}
(* Acyclic Tasks restrict the key of the dependencies to be strictly smaller than
 * the target key. We will fix the type of keys of acyclic tasks to natural numbers for clarity.*)
Definition AcyclicTasks (C : (Type -> Type) -> Type) :=
  forall (k : nat), Maybe (Task C (Fin.t k) nat).
\end{minted}
\vspace{-1mm}
\begin{minted}{coq}
(* Given a acyclic task description, a target key, and a store, the acyclic build
 * system returns a new store and is guarantied to terminates. *)
Definition AcyclicBuild (C : (Type -> Type) -> Type) (I : Type) :=
  AcyclicTasks C -> K -> Store I K V -> Store I K V.
\end{minted}
\vspace{-3mm}
\caption{Definitions of key build systems abstractions.}\label{fig-defs}
\vspace{-5mm}
\end{figure*}

% \vspace{-3mm}
% \subsection{The Task abstraction}

Building on the abstractions from~\cite{Mokhov2018icfp}, we model \hs{Store} as
an abstract data type parameterised by the type of keys~\hs{K}, values~\hs{V},
and build information~\hs{I} (see the Fig.~\ref{fig-defs}). A \hs{Task}
describes how to compute a value of a key (\hs{F V}), given a way to find values
of its dependencies (the callback \hs{K -> F V}). The type of effect~\hs{F}
associated with computing values is constrained by~\hs{C}, which emulates
Haskell's kind \hs{Constraint} and can be instantiated with concrete constraints,
such as \hs{Functor}, \hs{Applicative} or \hs{Monad}.

A \hs{Tasks} associates a task to a ~\emph{target key}.
If the target key is associated to an~\emph{input}
value --- there is~\hs{Nothing} to be done. Otherwise, the~\hs{run} function
contains a callback of type \hs{K -> F V} (conventionly called~\hs{fetch}),
which, given a key, produces a value and possibly couses some side-effects.
These side effects are of immense value here and to have flexible control over them
is the exact reason to keep \hs{F} polymorphic.

The polymorphic \hs{F} is the corner stone of the Build Systems \`a la Carte
framework: given a~\emph{single} task description, we want to explore~\emph{many}
build systems that can build it, exploiting distinctive features of various
type constructors \hs{F}. In the number of cases, \hs{F} will be instantiated
with \hs{State s} for some build-state \hs{s}, or it may become the
\hs{Const K} datatype which allows to statically calculate the dependencies of certain
types of \hs{Task}'s.

As an example, consider a description of the tasks to compute Fibonacci numbers:

\vspace{-1mm}
\begin{minted}{coq}
Definition fibonacci : Tasks Applicative nat nat :=
  fun n =>
  match n with
  | 0  => Nothing
  | 1  => Nothing
  | S (S m) => Just {|
               run := fun _ _ => fun fetch =>
                      Nat.add <$> fetch (S m)
                              <*> fetch m |}
  end.
\end{minted}
\vspace{-1mm}

The keys~\coq{0} and~\coq{1} correspond to inputs, thus no dependencies need to
be fetched. The rest of the keys are associated to non-trivial tasks, that fetch
the two previous Fibonacci numbers and return their sum.

Dependencies of applicative tasks can be calculated statically, without proving
an actual instance of~\hs{Store}:

\vspace{-1mm}
\begin{minted}{coq}
Definition dependencies {K V : Type}
  (task : Task Applicative K V) :
  list K := getConst
    ((run task) (fun k => mkConst (cons k nil))).
\end{minted}
\vspace{-1mm}

The~\coq{dependencies} function instantiates the~\hs{F} type constructor
with~\hs{Const (list K)} and executes the task with a~\hs{fetch} callback that
just reports the fetched key. The keys then get accumulated to a list by the
\hs{Applicative} instance of~\hs{Const K} datatype.

The~\coq{fibonacci} task description is applicative, therefore it can be
statically analysed for dependencies, and the correctness of the dependency
function can be trivially proved for this specific case:

\vspace{-1mm}
\begin{minted}{coq}
Theorem deps_fib_correct : forall n,
  deps_fib (S (S n)) = (S n :: n :: nil).
Proof. reflexivity. Qed.
\end{minted}
\vspace{-1mm}

Task descriptions trap the recursion inside the fetching callback. In case of
the~\coq{dependencies} function, fetching callback is not recursive, thus the
dependencies ``build system'' always terminates. However, proper build systems
described in~\cite{Mokhov2018icfp} require a recursive fetching callback.
At this point, we have not managed to encode even the simples build
system~\hs{busy}, because Coq's termination checker is a hard judge to persuade.
To tackle this issue, we attempt to formalise the notion of always-terminating
acyclic task descriptions.

\vspace{-3mm}
\subsection{Acyclic Tasks}

We attempt to modify the build systems framework and encode the always terminating
task descriptions as the~\hs{AcyclicTasks} datatype. In contrasts with the regular,
possibly non-terminating \hs{Tasks}, acyclic task descriptions are represented as
a dependent function: the type of keys depends on the value of the target key.
Thus, every call of the \hs{fetch} callback can only be performed with a smaller key.
The Fibonacci function can also be represented as an acyclic task:

\vspace{-1mm}
\begin{minted}{coq}
Definition fibonacci :
  AcyclicTasks Applicative := fun n =>
  match n with
  | 0  => Nothing
  | 1  => Nothing
  | S (S m) => Just
      {| run := fun _ _ => fun fetch =>
           Nat.add <$> fetch (from_nat (S m))
                   <*> fetch (inject1 (from_nat m)) |}
  end.
\end{minted}
\vspace{-1mm}

\vspace{-3mm}
\section{Discussion}\label{sec-discussion}
\vspace{-1mm}

The Haskell-embedded framework for modelling build systems presented in the
paper~\cite{Mokhov2018icfp} exploits GHC's~\emph{Rank2Types}
and~\emph{ConstraintKinds} to achieve high level of brevity and code reuse.
Coq's type system is reach enough to emulate these features and we have been able
to define all the required abstractions. However, since Coq is a total language,
defining even the simplest build system~\hs{busy} requires a lot more
effort than anticipated.

Even restricting the~\hs{Tasks} abstraction to
supposingly always terminating~\hs{AcyclicTasks} has not made the problem easier.

The main take-away from this experience for us is the necessity to rigorously
prove the totality of a build system before even attempting to prove its correctness.

% %% Acknowledgments
% \begin{acks}                            %% acks environment is optional

% \end{acks}


%% Bibliography
\bibliography{biblio}

% %% Appendix
% \appendix
% \section{Appendix}

% Text of appendix \ldots

\end{document}
