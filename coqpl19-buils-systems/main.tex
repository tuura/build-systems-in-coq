%% For double-blind review submission, w/o CCS and ACM Reference (max submission space)
% \documentclass[sigplan,review,anonymous]{acmart}\settopmatter{printfolios=true,printccs=false,printacmref=false}
%% For double-blind review submission, w/ CCS and ACM Reference
%\documentclass[sigplan,review,anonymous]{acmart}\settopmatter{printfolios=true}
% For single-blind review submission, w/o CCS and ACM Reference (max submission space)
\documentclass[sigplan,review]{acmart}\settopmatter{printfolios=true,printccs=false,printacmref=false}
%% For single-blind review submission, w/ CCS and ACM Reference
%\documentclass[sigplan,review]{acmart}\settopmatter{printfolios=true}
%% For final camera-ready submission, w/ required CCS and ACM Reference
%\documentclass[sigplan]{acmart}\settopmatter{}

\usepackage{bookmark}
\usepackage{booktabs}
\usepackage{subcaption}
\usepackage[utf8]{inputenc}
\usepackage[T1]{fontenc}
\usepackage{xspace}
\usepackage{fancyhdr}

% Haskell code snippets and useful shortcuts
\usepackage{minted}
\setminted[coq]{escapeinside=@@}
\newcommand{\hs}{\mintinline{haskell}}
\newcommand{\coq}{\mintinline{coq}}
\newcommand{\cmd}[1]{\textsf{\color[rgb]{0,0,0.5} #1}}
\newcommand{\teq}{\smaller $\sim$}
\newcommand{\ghci}{$\lambda$>}
\newcommand{\defeq}{\stackrel{\text{def}}{=}}
\newcommand{\std}[1]{{\color[rgb]{0,0.3,0} #1}}
\newcommand{\blk}[1]{{\color[rgb]{0,0,0} #1}}

\newcommand{\Bazel}{\textsc{Bazel}\xspace}
\newcommand{\Buck}{\textsc{Buck}\xspace}
\newcommand{\Calc}{\textsc{Calc}\xspace}
\newcommand{\Cloud}{\textsc{Cloud}\xspace}
\newcommand{\CloudBuild}{\textsc{CloudBuild}\xspace}
\newcommand{\Dune}{\textsc{Dune}\xspace}
\newcommand{\Excel}{\textsc{Excel}\xspace}
\newcommand{\Fabricate}{\textsc{Fabricate}\xspace}
\newcommand{\Incremental}{\textsc{Incremental}\xspace}
\newcommand{\Make}{\textsc{Make}\xspace}
\newcommand{\Ninja}{\textsc{Ninja}\xspace}
\newcommand{\Nix}{\textsc{Nix}\xspace}
\newcommand{\Pluto}{\textsc{Pluto}\xspace}
\newcommand{\Redo}{\textsc{Redo}\xspace}
\newcommand{\Reflow}{\textsc{Reflow}\xspace}
\newcommand{\Shake}{\textsc{Shake}\xspace}
\newcommand{\Tup}{\textsc{Tup}\xspace}
\newcommand{\store}{\hs{k}~\hs{->}~\hs{v}\xspace}
\newcommand{\storef}{\hs{k}~\hs{->}~\hs{f}~\hs{v}\xspace}

%% Conference information
%% Supplied to authors by publisher for camera-ready submission;
%% use defaults for review submission.
\acmConference[CoqPL'19]{The Fifth International Workshop on Coq for Programming Languages}{January 19, 2019}{Cascais/Lisbon, Portugal}
\acmYear{2019}
\acmISBN{} % \acmISBN{978-x-xxxx-xxxx-x/YY/MM}
\acmDOI{} % \acmDOI{10.1145/nnnnnnn.nnnnnnn}
\startPage{1}

%% Copyright information
%% Supplied to authors (based on authors' rights management selection;
%% see authors.acm.org) by publisher for camera-ready submission;
%% use 'none' for review submission.
\setcopyright{none}
%\setcopyright{acmcopyright}
%\setcopyright{acmlicensed}
%\setcopyright{rightsretained}
%\copyrightyear{2018}           %% If different from \acmYear

%% Bibliography style
\bibliographystyle{ACM-Reference-Format}
%% Citation style
%\citestyle{acmauthoryear}  %% For author/year citations
%\citestyle{acmnumeric}     %% For numeric citations
%\setcitestyle{nosort}      %% With 'acmnumeric', to disable automatic
                            %% sorting of references within a single citation;
                            %% e.g., \cite{Smith99,Carpenter05,Baker12}
                            %% rendered as [14,5,2] rather than [2,5,14].
%\setcitesyle{nocompress}   %% With 'acmnumeric', to disable automatic
                            %% compression of sequential references within a
                            %% single citation;
                            %% e.g., \cite{Baker12,Baker14,Baker16}
                            %% rendered as [2,3,4] rather than [2-4].


%%%%%%%%%%%%%%%%%%%%%%%%%%%%%%%%%%%%%%%%%%%%%%%%%%%%%%%%%%%%%%%%%%%%%%
%% Note: Authors migrating a paper from traditional SIGPLAN
%% proceedings format to PACMPL format must update the
%% '\documentclass' and topmatter commands above; see
%% 'acmart-pacmpl-template.tex'.
%%%%%%%%%%%%%%%%%%%%%%%%%%%%%%%%%%%%%%%%%%%%%%%%%%%%%%%%%%%%%%%%%%%%%%


%% Some recommended packages.
\usepackage{booktabs}   %% For formal tables:
                        %% http://ctan.org/pkg/booktabs
\usepackage{subcaption} %% For complex figures with subfigures/subcaptions
                        %% http://ctan.org/pkg/subcaption


\begin{document}

%% Title information
\title[]{A Coq Formalisation of Build Systems}         %% [Short Title] is optional;
                                        %% when present, will be used in
                                        %% header instead of Full Title.
% \titlenote{with title note}             %% \titlenote is optional;
                                        %% can be repeated if necessary;
                                        %% contents suppressed with 'anonymous'
\subtitle{Experience report}                     %% \subtitle is optional
% \subtitlenote{with subtitle note}       %% \subtitlenote is optional;
                                        %% can be repeated if necessary;
                                        %% contents suppressed with 'anonymous'

%% Author information
%% Contents and number of authors suppressed with 'anonymous'.
%% Each author should be introduced by \author, followed by
%% \authornote (optional), \orcid (optional), \affiliation, and
%% \email.
%% An author may have multiple affiliations and/or emails; repeat the
%% appropriate command.
%% Many elements are not rendered, but should be provided for metadata
%% extraction tools.

\author{Georgy Lukyanov}
\affiliation{
  \department{School of Engineering}
  \institution{Newcastle University}
  \city{Newcastle upon Tyne}
  \country{United Kingdom}
}
\email{g.lukyanov2@ncl.ac.uk}

\author{Andrey Mokhov}
\affiliation{
  \department{School of Engineering}
  \institution{Newcastle University}
  \city{Newcastle upon Tyne}
  \country{United Kingdom}
}
\email{andrey.mokhov@ncl.ac.uk}

%% Abstract
%% Note: \begin{abstract}...\end{abstract} environment must come
%% before \maketitle command
\begin{abstract}
The gene pool of software build systems and various incremental computation
frameworks is becoming richer every day. These complex build systems and
frameworks use subtle algorithms and are mission-critical, yet to the best of
our knowledge there is no machine-checked validation of their correctness.

A recent ICFP paper ``Build Systems \`a la Carte'' presented a definition of
correctness for build systems, and modelled several major build systems in
Haskell, without exhibiting any proof of their correctness. We build on this
work by translating the Haskell abstractions to Coq and making the necessary
adjustments to capture the notion of build task acyclicity, which is essential
for proving termination.

This is an experience report on on-going work which is very far from being
complete. We present our motivation and key abstractions developed so far, and
seek feedback and collaboration from the Coq community.
\end{abstract}


%% 2012 ACM Computing Classification System (CSS) concepts
%% Generate at 'http://dl.acm.org/ccs/ccs.cfm'.
\begin{CCSXML}
<ccs2012>
<concept>
<concept_id>10011007.10011006.10011008</concept_id>
<concept_desc>Software and its engineering~General programming languages</concept_desc>
<concept_significance>500</concept_significance>
</concept>
<concept>
<concept_id>10003456.10003457.10003521.10003525</concept_id>
<concept_desc>Social and professional topics~History of programming languages</concept_desc>
<concept_significance>300</concept_significance>
</concept>
</ccs2012>
\end{CCSXML}

\ccsdesc[500]{Software and its engineering~General programming languages}
\ccsdesc[300]{Social and professional topics~History of programming languages}
%% End of generated code


%% Keywords
%% comma separated list
\keywords{keyword1, keyword2, keyword3}  %% \keywords are mandatory in final camera-ready submission


%% \maketitle
%% Note: \maketitle command must come after title commands, author
%% commands, abstract environment, Computing Classification System
%% environment and commands, and keywords command.
\maketitle


\section{Introduction}

\section{The key abstractions in Coq}

\begin{figure*}[t]
\begin{minted}{coq}
(* An abstract store containing a key/value map and persistent build information *)
Inductive Store (I K V : Type) := mkStore { info : I; values : K -> V}.

(* Task describes how to build values of type V from keys of type K *)
Inductive Task (C : (Type -> Type) -> Type) (K V : Type) := {
  run : forall {F} `{CF: C F}, (K -> F V) -> F V}.

Definition Tasks (C : (Type -> Type) -> Type) (K V : Type) :=
  K -> Maybe (Task C K V).

(* Given a task description, a target key, and a store, the build
 * system returns a new store in which the value of the target key is up to date. *)
Definition Build (C : (Type -> Type) -> Type) (I K V : Type) :=
  Tasks C K V -> K -> Store I K V -> Store I K V.

(* A total Tasks restrict the key of the dependencies to be strictly
* smaller than the top-level one. In this abstract we will fix the type of
* keys of total tasks to natural numbers for clarity.*)
Definition TotalTasks (C : (Type -> Type) -> Type) :=
  forall (k : nat), Maybe (Task C (Fin.t k) nat).

(* Given a total task description, a target key, and a store, the total build
 * system returns a new store and is guarantied to terminates. *)
Definition TotalBuild (C : (Type -> Type) -> Type) (I : Type) :=
  TotalTasks C -> K -> Store I K V -> Store I K V.
\end{minted}
\caption{Definitions of key build systems abstractions.}\label{fig-defs}
\end{figure*}

\subsection{Components of a build system} 

\paragraph{Keys, values and the store.}
Build systems are essentialy functions that transform a key/value store to
make it up to date. In software build systems the store is the file system,
the keys are filenames, and the values are file contents. In Excel, the store is
the worksheets, the keys are cell names (such as A1) and the values are numbers,
strings, etc., displayed as the cell contents.

\paragraph{Input, output and intermediate values.}
Some values must be provided by
the user as \emph{input}. For example, \cmd{main.c} can be edited by the user
who relies on the build system to compile it into \cmd{main.o} and subsequently
\cmd{main.exe}. End build products, such as \cmd{main.exe}, are \emph{output}
values. All other values (in this case \cmd{main.o}) are \emph{intermediate};
they are not interesting for the user but are produced in the process of turning
inputs into outputs.

\paragraph{Persistent build information.} As well as the key/value mapping, the
store also contains information maintained by the build system itself, which
persists from one invocation of the build system to the next -- its ``memory''.

\paragraph{Task description.} Any build system requires the user to specify how
to compute the new value for one key, using the (up to date) values of its
dependencies. We call this specification the \emph{task description}. For
example, in \Excel, the formulae of the spreadsheet constitute the task
description; in \Make the rules in the makefile are the task description.

\paragraph{Build system.} A \emph{build system} takes a task description, a
\emph{target} key, and a store, and returns a new store in which the target key
and all its dependencies have an up to date value.

\subsection{The Task abstraction}

In the Build Systems \`a la Carte framework, the build scenarios are abstracted
as \hs{Task}'s (see the figure~\ref{fig-defs}). A \hs{Task}
describes how to build a value of type \hs{V} from a key of type \hs{K}.
The argument \hs{C} here emulates Haskell's kind \hs{Constraint} and can be
instantiated with concrete typeclass constraints such as \hs{Functor},
\hs{Applicative} or \hs{Monad}.

Omiting the type variable bindings, the task description is a~\emph{target key}
and a task~\emph{semantics}. If the target key is associated to an~\emph{input}
value --- there is~\coq{Nothing} to be done. Otherwise, the~\coq{run} function
contains the~\coq{fetch} callback of type \coq{K -> F V}, which, given a key,
produces a value and possibly couses some side-effects. These side effects are
of immense value and to have flexible control over them is the exact reason to
keep \coq{F} polymorphic.

The \hs{run} field is parametrised by a
polymorphic type constructor \hs{F} which is restricted by the constraint \hs{C}.
The polymorphic \hs{F} is the corner stone of the Build Systems \`a la Carte
framework: given a~\emph{single} task description, we want to explore~\emph{many}
build systems that can build it, exploiting distinctive features of various
type constructors \hs{F}. In the number of cases, \hs{F} will be instantiated
with \hs{State s} for some build-state \hs{s}, or it may become the
\hs{Const K} datatype which allows to statically calculate the dependencies of certain
types of \hs{Task}'s.

\subsection{The Build abstraction}

\subsection{Total Tasks}


% %% Acknowledgments
% \begin{acks}                            %% acks environment is optional

% \end{acks}


%% Bibliography
\bibliography{biblio.bib}

% %% Appendix
% \appendix
% \section{Appendix}

% Text of appendix \ldots

\end{document}
